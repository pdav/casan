%
% ICube - Network research group meeting, 28/01/2013
%

\documentclass [12pt] {beamer}

    \usepackage [utf8] {inputenc}
    \usepackage {helvet}

    \usepackage {graphicx}
    % \usepackage {tikz}

    % \usepackage {hyperref}
    \hypersetup
    {
	pdfauthor={Pierre David},
        pdfsubject={A pragmatic architecture for sensor networks},
	pdftitle={A pragmatic architecture for sensor networks},
	pdfkeywords={REST, HTTP, CoRE, CoAP, Bonjour}
    }


    \usetheme {default}
    % \useoutertheme [headline=empty,footline=authortitle] {miniframes}
    \usefonttheme {structurebold}

    \setbeamercolor {frametitle} {fg=red}
    \setbeamerfont {frametitle} {series=\bfseries}
    \setbeamertemplate {frametitle} {
	\textbf {\insertframetitle} \par
    }

    \pgfdeclareimage[height=0.5cm]{logo}{icube}
    % \logo{\pgfuseimage{logo}}

    \setbeamertemplate {footline} {%
	\vskip 1ex
	    \hskip 1ex
	    % \insertlogo
	    \pgfuseimage {logo}
	    \hfill
	    \insertframenumber/\inserttotalframenumber
	    \hspace* {1ex}
	\vskip 1ex
    }

    \setbeamertemplate {navigation symbols} {}

    \title {A pragmatic architecture \\
	 	for sensor networks}
    \author {Pierre David \\ \texttt {pda@unistra.fr}}
    \institute {ICube -- Network Research Group}
    \date [Network Research Group meeting] {13 february 2013}

    \newcommand {\imply} {$\Rightarrow$\ }

\begin {document}

\begin {frame} [plain]
    \titlepage
\end {frame}

%%%%%%%%%%%%%%%%%%%%%%%%%%%%%%%%%%%%%%%%%%%%%%%%%%%%%%%%%%%%%%%%%%%%%%%%%%%%%%
\section* {Outline}

\begin {frame} {Outline}
    \tableofcontents
\end {frame}

%%%%%%%%%%%%%%%%%%%%%%%%%%%%%%%%%%%%%%%%%%%%%%%%%%%%%%%%%%%%%%%%%%%%%%%%%%%%%%
\section {The REST architecture}

\begin {frame} {SOS control messages}
    \begin {center}
    \includegraphics [width=.7\textwidth,keepaspectratio] {handshake}
    \end {center}
\end {frame}

\begin {frame} {Slave automaton}
    \includegraphics [width=.98\textwidth,keepaspectratio] {auto-slave}

\end {frame}

\begin {frame} {Master automaton}
    \includegraphics [width=.98\textwidth,keepaspectratio] {auto-master}

\end {frame}

\end {document}

\begin {frame} {Software view point}
    Current trend is to use a RESTful architecture:

    \begin {columns} [c]
	\begin {column} [c] {.5\textwidth}
	    \begin {center}
		\includegraphics [height=5cm,keepaspectratio] {stack}
	    \end {center}
	\end {column}
	\begin {column} [c] {.5\textwidth}
	    What is REST?
	\end {column}
    \end {columns}
    \vspace {3mm}

    This is \alert {not} the standard OSI 7 layer model
\end {frame}

\begin {frame} {The REST architecture}
    REST = Representational State Transfer

    \begin {itemize}
	\item formally defined in 2000 (Roy Fielding)
	\item REST is an architecture, not a protocol (... but...)
    \end {itemize}
%
%    \pause
%    Components of a REST architecture:
%    \begin {itemize}
%	\item resources
%	\item representation of resources
%	\item manipulation of resources
%	\item self-descriptive messages
%	\item hypermedia as the engine of application state
%    \end {itemize}
\end {frame}

\begin {frame} {The REST architecture constraints}

    \includegraphics<1> [width=\textwidth,keepaspectratio] {rest-princ0}
    \includegraphics<2> [width=\textwidth,keepaspectratio] {rest-princ1}
    \includegraphics<3> [width=\textwidth,keepaspectratio] {rest-princ2}
    \includegraphics<4> [width=\textwidth,keepaspectratio] {rest-princ3}
    \includegraphics<5> [width=\textwidth,keepaspectratio] {rest-princ4}
    \includegraphics<6> [width=\textwidth,keepaspectratio] {rest-princ5}
    \includegraphics<7> [width=\textwidth,keepaspectratio] {rest-princ6}
    \includegraphics<8> [width=\textwidth,keepaspectratio] {rest-princ7}
    \includegraphics<9> [width=\textwidth,keepaspectratio] {rest-princ}

\end {frame}

\begin {frame} {The REST architecture}
    Application to HTTP:
    \begin {itemize}
	\item resource id: \\
	    \imply URL
	\item representation of a resource: \\
	    \imply JSON or XML or text formats
	\item manipulation of resources: HTTP operations \\
	    \imply GET, PUT, DELETE and POST
	\item meta-data: HTTP headers \\
	    \imply data types \\
	    \imply cache control
    \end {itemize}

    \pause
    REST is an architecture not a protocol... \\
    \imply but REST is often seen
    as an alternative to other RPC mechanisms (e.g. SOAP)
\end {frame}

\begin {frame} {Example: delicious.com}
    \url {http://delicious.com} is a social bookmarking service:
    \begin {itemize}
	\item store my Web bookmarks in the cloud
	\item tag my Web bookmarks
	\item share my Web bookmarks with other people
	\item discover other's Web bookmarks
    \end {itemize}
    \begin {columns} [c,onlytextwidth]
	\begin {column} [c] {.6\textwidth}
	    \begin {center}
		\includegraphics [width=.9\textwidth,keepaspectratio]{delicious}
	    \end {center}
	\end {column}
	\pause
	\begin {column} [c] {.4\textwidth}
	    Delicious has a RESTful API
	\end {column}
    \end {columns}
\end {frame}

\begin {frame} {Example: delicious.com}
    To get my public bookmarks tagged with "histinfo":

    {
	\small
	\hspace* {5mm}
	\alert {GET http://feeds.delicious.com/v2/json/pdagog/histinfo}
    }
    \vspace* {3mm}

    \pause
    Result (JSON format): \\
    \begin {quote}
	\scriptsize
	[ \\
	    \hspace* {3mm} \{ \\
		\hspace* {6mm} "a": "pdagog", \\
		\hspace* {6mm} "d": "Google60", \\
		\hspace* {6mm} "u": "http://www.masswerk.at/google60/", \\
		\hspace* {6mm} "t": ["histinfo", "web", "humour"] \\
	    \hspace* {3mm} \}, \\
	    \hspace* {3mm} \{ \\
		\hspace* {6mm} "a": "pdagog", \\
		\hspace* {6mm} "d": "Unix Preservation society", \\
		\hspace* {6mm} "u": "http://minnie.tuhs.org/", \\
		\hspace* {6mm} "t": ["freebsd", "unix", "histinfo"] \\
	    \hspace* {3mm} \}, ... \\
	]
    \end {quote}

%    \pause
%    \begin {itemize}
%	\item See https://delicious.com/developers
%	\item Some resources need an authentication
%    \end {itemize}
\end {frame}

\begin {frame} {Application to sensor networks}
    \begin {center}
	\includegraphics [width=\textwidth,keepaspectratio] {rest-sens}
    \end {center}

\end {frame}

%%%%%%%%%%%%%%%%%%%%%%%%%%%%%%%%%%%%%%%%%%%%%%%%%%%%%%%%%%%%%%%%%%%%%%%%%%%%%%
\section {Towards a pragmatic architecture}

\begin {frame} {Outline}
    \tableofcontents [current]
\end {frame}

\begin {frame} {Towards a pragmatic architecture}

    Each sensor is an autonomous node. It must embed an HTTP server over
    a complete network stack.

    \begin {columns} [c,onlytextwidth]
	\begin {column} [c] {.3\textwidth}
	    \begin {center}
		\includegraphics [width=.8\textwidth,keepaspectratio] {stack}
	    \end {center}
	\end {column}
	\begin {column} [c] {.7\textwidth}
	    \pause
	    ... and a lot of other protocols:
	    \begin {itemize}
		\item SSL for confidentiality
		\item Bonjour for service discovery
		\item multicast support for Bonjour
		\item DNS for naming and DNS-SD
		\item and many other things...
	    \end {itemize}
	    \vspace* {2mm}
	    \imply 3 decades of associated protocols
	\end {column}
    \end {columns}
\end {frame}

\begin {frame} {Towards a pragmatic architecture}

    With such a stack:

    \begin {itemize}
	\item sensors need CPU power
	\item sensors need energy
	\item sensor software is complex
	\item sensor software has bugs
	\item sensors need software updates
    \end {itemize}

    \pause
    \imply manufacturing and maintenance cost

    \pause
    \begin {columns} [c,onlytextwidth]
	\begin {column} [t] {.9\textwidth}
	    Can we embed such a stack into a lightbulb?

	    Can we afford the costs for all lightbulbs?
	\end {column}
	\begin {column} [c] {.1\textwidth}
	    \begin {center}
		\includegraphics [width=.9\textwidth,keepaspectratio] {lightbulb}
	    \end {center}
	\end {column}
    \end {columns}

    \pause
    \imply no...

\end {frame}

\begin {frame} {Towards a pragmatic architecture}

    Can we simplify this architecture?

    \pause
    \imply yes

    \pause
    \vspace* {2mm}

    Central idea:
    \begin {itemize}
	\item reduce sensor intelligence
	\item improve intelligence of WSN sink (or similar) in
	    order to make a high-level gateway
    \end {itemize}

    \vspace* {2mm}

    \pause
    Make this gateway run a \alert {Sensor Operating System} (SOS)

    \begin {itemize}
	\item SOS is a distributed operating system
	\item gateway is the \alert {master} processor
	\item sensor \alert {nodes} are viewed as peripheral devices
	\item sensor operations are REST operations on the SOS
    \end {itemize}

    \imply simplify protocol between master and nodes

\end {frame}

\begin {frame} {Towards a pragmatic architecture}

    \includegraphics [width=\textwidth,keepaspectratio] {arch-req}

    \pause
    Master must:
    \begin {itemize}
	\item discover neighborhood
	\item map node addresses to REST resources
	\item respond to REST requests on these resources
	\item hide the sensor network from the Internet
    \end {itemize}
\end {frame}

\begin {frame} {Towards a pragmatic architecture}
    \includegraphics [width=\textwidth,keepaspectratio] {arch-req}

    Master-Sensor protocol does not need:
    \begin {itemize}
	\item long-distance optimizations
	    \imply no TCP layer
	\item inter-network routing
	    \imply no IP layer
	\item multiple applications on sensor
	    \imply no UDP port
	\item inter-sensor communication (only master-sensor)

    \end {itemize}
    \pause
    \vspace* {1mm}
    Need for a simplified protocol: SOS protocol

\end {frame}

\begin {frame} {Towards a pragmatic architecture}
    \includegraphics [width=\textwidth,keepaspectratio] {arch-stack}

    \vspace* {2mm}
    \small
    Master~: Unix-like (Linux, BSD) \\
    Nodes~: micro-controller based OS (Arduino, TinyOS, Contiki)

\end {frame}

\begin {frame} {SOS protocol requirements}
    Master, nodes and sensors:
    \begin {itemize}
	\item a single node may have multiple sensors 
	    \includegraphics [width=10mm,keepaspectratio] {power-outlet2}
	\item nodes and sensors are named \\
	    power-outlet25, unit1
	\item master must auto-discover nodes and sensors
	\item mutual authentication between master and nodes
	\item REST mapping: resource names and operations \\
	    http://power-outlet25.my.home/unit1/switch?val=on
	\item redundancy: multiple master support
    \end {itemize}
\end {frame}

\begin {frame} {SOS protocol requirements}
    Traffic handling:
    \begin {itemize}
	\item scalable \\
	    all lightbulbs in TPS, C building, 2nd floor
	\item block transfer (e.g. cameras)
	\item traffic congestion should be regulated by master \\
	    camera2, you can use at most 640x480 \\
	    (not sure about that)
    \end {itemize}
\end {frame}

\begin {frame} {SOS protocol requirements}
    Data link layer interface:

    \begin {itemize}
	\item data link layer agnostic (802.15.4, PLC, etc.)
	\item allow use of a multi-hop data link layer
	\item allow small packet sizes
    \end {itemize}

\end {frame}

\begin {frame} {Opportunities}
    Additional benefits of intelligence on master:

    \begin {itemize}
	\item simulate content-centric sensor networks \\
	    (e.g. no need for TinyDB or similar)
	\item aggregate sensor information \\
	    (e.g.  raise a condition if more than $x$ sensors report
		    an high temperature)
	\item cache information and reduce traffic \\
	    (especially when more than one client)
	\item etc.

    \end {itemize}

\end {frame}

\begin {frame} {Summary}
    Network stack required by REST is too complex, and too
    costly to deploy everywhere.
    \vspace* {5mm}

    Idea: design the Sensor Operating System:
    \begin {itemize}
	\item sensors are devices of a master processor
	\item master-sensor protocol must be very light
	\item yet, sensors are "on the Internet" \\
	    even if they have no IP address
    \end {itemize}

    \pause
    Analogy:
    \begin {itemize}
	\item Mobile IP is too complex for mobile nodes
	\item Current trend is to shift mobility support from nodes to infrastructure
	\item \imply Proxy MIP has been preferred to Mobile IP
    \end {itemize}
\end {frame}

%%%%%%%%%%%%%%%%%%%%%%%%%%%%%%%%%%%%%%%%%%%%%%%%%%%%%%%%%%%%%%%%%%%%%%%%%%%%%%
\section {Related work: CoRE}

\begin {frame} {Outline}
    \tableofcontents [current]
\end {frame}

%\begin {frame} {Related work}
%    IETF remains in the "IP everywhere" dogma (surprise?)
%
%    \vspace* {2mm}
%
%    Related works:
%    \begin {itemize}
%	\item discovery protocols (Web Linking, SLP, UPNP, Bonjour)
%	\item 6LoWPAN
%	\item \alert {CoRE} and \alert {CoAP}
%    \end {itemize}
%
%\end {frame}

\begin {frame} {CoRE / CoAP}
    CoRE = Constrained RESTful Environments

    \vspace* {2mm}

    CoRE is a framework for REST-architectured applications on constrained
    IP networks.

    \begin {itemize}
	\item young IETF working group
	\item first meeting: march 2010
	\item \~{} 20 internet drafts (wg and related)
	\item one published RFC (6690)
    \end {itemize}

    CoAP (Constrained Application Protocol) is the protocol defined by
    the CoRE working group.

\end {frame}

\begin {frame} {CoRE / CoAP}
    In short: CoAP is HTTP over UDP.

    \vspace* {3mm}

    \includegraphics [width=\textwidth,keepaspectratio] {core-stack}

    \vspace* {3mm}
    CoAP may use standard UDP/IP stack on classical networks as well as
    6LoWPAN on constrained networks \\
    CoAP may also use DTLS (Datagram Transport Layer Security, RFC 6347 --
    TLS over UDP), or IPSec.

\end {frame}

\begin {frame} {CoRE / CoAP}
    Architectural view:

    \includegraphics [width=\textwidth,keepaspectratio] {core-arch}

    \pause
    \begin {itemize}
	\item "IP everywhere" dogma: each sensor must be IP addressable
	    from the Internet

	\item mapping from HTTP to CoAP on the proxy

    \end {itemize}
\end {frame}

\begin {frame} {CoRE / CoAP}

    \begin {itemize}
	\item 
	    IMHO: end-to-end connection is unrealistic in the real world

	\item
	    \pause
	    So, why keep an UDP/IP stack on the constrained network?

	\item
	    \pause
	    However, CoAP has some good features. Let's see...

    \end {itemize}

\end {frame}

\begin {frame} {CoAP -- Message format}
    \includegraphics [width=\textwidth,keepaspectratio] {coap-fmt}

    \pause
    \begin {itemize}
	\item Type: transaction type
	    \begin {itemize}
		\item CON (Confirmable): issuer expects an answer
		\item NON (Non-Confirmable): no answer needed
		\item ACK (Acknowledgment)
		\item RST (Reset)
	    \end {itemize}
	\item Code: request (GET, PUT, etc.) or return code (205, 404, etc.)
	\item Options: list of TLVs (type-length-value)
    \end {itemize}
\end {frame}

\begin {frame} {CoAP -- A typical request}
    \includegraphics [width=\textwidth,keepaspectratio] {coap-fmt}
    \vspace* {2mm}
    \includegraphics [width=\textwidth,keepaspectratio] {coap-req-con}

    \vspace* {2mm}
    No acknowledgment: client retransmits after a timeout

\end {frame}

\begin {frame} {CoAP -- A typical request}
    \begin {itemize}
	\item Message Id: to match each request and response messages
	\item Token: to match responses independently from the
	    underlying messages
    \end {itemize}

    \includegraphics [width=\textwidth,keepaspectratio] {coap-req-tok}

\end {frame}

\begin {frame} {CoAP -- Another typical request}
    \includegraphics [width=\textwidth,keepaspectratio] {coap-fmt}
    \vspace* {2mm}
    \includegraphics [width=\textwidth,keepaspectratio] {coap-req-non}

    \vspace* {2mm}
    No acknowledgment expected from the client

\end {frame}

\begin {frame} {CoAP -- Caching}
    \begin {itemize}
	\item Cacheability is determined by the response code \\
	    \vspace* {1mm}
	    Examples:
	    \begin {tabular} {|l|l|l|} \hline
		201 & Created & \alert {no} \\ \hline
		205 & Content & \alert {yes} \\ \hline
		4xx & Client error & \alert {yes} \\ \hline
	    \end {tabular}
	    \vspace* {1mm}
	\item Option \alert {Max-Age}: expiration time given by the server
	\item Option \alert {Etag}: entity tag supplied by the server \\
	    Client uses it to re-validate an expired answer
    \end {itemize}

\end {frame}

\begin {frame} {CoAP -- Block transfers (aka TFTP)}
    \begin {itemize}
	\item Option \alert {Block1} : for large requests
	\item Option \alert {Block2} : for large responses
    \end {itemize}
    \includegraphics [width=\textwidth,keepaspectratio] {coap-req-blk}

\end {frame}

\begin {frame} {CoAP -- Observe}
    A CoAP client can subscribe to a resource with the \alert {Observe}
    option. Each time the observed resource changes, a notification is
    sent to the client.

    \vspace* {2mm}

    \includegraphics [width=\textwidth,keepaspectratio] {coap-req-obs}

    \vspace* {2mm}

    \pause
    End of observation : \alert {Max-Age} expired, RST, CON without ACK,
    or new GET without \alert {Observe} option.

\end {frame}

\begin {frame} {CoAP -- Multicast}
    CoAP may use multicast addresses:

    \begin {quote}
	\small coap://all-lightbulbs-flr2-bdgC.tps.u-strasbg.fr/status
    \end {quote}

    Special considerations:
    \begin {itemize}
	\item Non confirmable requests only
	\item Idempotent requests only (e.g. GET, but not POST)
	\item Configuration of group membership: with unicast CoAP \\
	    \imply {\small POST /gp?n=all-light...\&ip=ff05::4200:f7fe:ed37:14ca}
	    \\
	    Receiving node will subscribe to the multicast group

    \end {itemize}
\end {frame}

\begin {frame} {CoRE -- Resource discovery}
    Which resources are available on an HTTP node?

    \begin {itemize}
	\item "Well-Known" interface (RFC 5785): \\
	    \hspace* {5mm}
	    http://server/\alert {.well-known}/xxx \\
	    xxx names are registered by IANA
	\item Web Linking (RFC 5988): relationship between resources \\
	    \hspace* {5mm}
	    {
		\footnotesize
		\alert {Link}:
		    $<$http://.../book/ch2$>$;
		    \alert {rel}="next";
		    title="next chapter"
	    }
	    \\
	    Link is an HTTP header \\
	    Relation types are registered by IANA \\
	    Web Linking may also be used in an HTML document
    \end {itemize}
\end {frame}

\begin {frame} {CoRE -- Resource discovery}
    Application to CoRE:

    \begin {itemize}
	\item CoRE link interface (RFC 6690): \\
	    \hspace* {5mm}
		coap://server/.well-known/\alert {core}
		\hspace* {4mm} (or http://)
	    \\
	    Default entry point for requesting the list of resources \\
	    \imply directory of related links (web-linking)

	\item Get the list of available resources: \\
	    \hspace* {5mm}
		\alert {GET} /.well-known/core
	    \\
	    May be filtered: GET /.well-known/core\alert {?rt=light-lux}

	\item Discover all resources on a CoRE network: \\
	    \hspace* {5mm}
		\alert {multicast (all CoAP nodes)} GET /.well-known/core

	\item Build a resource directory (e.g. on a proxy),
	    by registering a new resource: \\
	    \hspace* {5mm}
		\alert {POST} /.well-known/core
    \end {itemize}

\end {frame}

%%%%%%%%%%%%%%%%%%%%% METTRE L'ACCENT SUR LES PROTOCOLES NECESSAIRES AUTOUR : DNS, SSL, DTLS, Bonjour, 30 ans de protocole

\begin {frame} {Summary}
    CoAP:
    \begin {itemize}
	\item assumes an end-to-end IP connectivity
	\item needs a lot of protocols:
	    \begin {itemize}
		\item UDP, which needs IP
		\item 6LoWPAN, which needs IPv6
		\item RPL and MPL for routing
		\item Multicast and associated protocols
		\item DTLS
		\item ...
	    \end {itemize}
    \end {itemize}

    However:
    \begin {itemize}
	\item CoAP is a serious basis to build upon
	\item SOS = CoAP over Data-Link?
    \end {itemize}
\end {frame}

%%%%%%%%%%%%%%%%%%%%%%%%%%%%%%%%%%%%%%%%%%%%%%%%%%%%%%%%%%%%%%%%%%%%%%%%%%%%%%
\section {Proposed SOS architecture}

\begin {frame} {Outline}
    \tableofcontents [current]
\end {frame}

\begin {frame} {SOS architecture}
    \begin {center}
	\includegraphics<1> [width=\textwidth,keepaspectratio] {sos-arch11}
	\includegraphics<2> [width=\textwidth,keepaspectratio] {sos-arch12}
	\includegraphics<3> [width=\textwidth,keepaspectratio] {sos-arch13}
    \end {center}
\end {frame}

\begin {frame} {SOS architecture}
    \begin {center}
	\includegraphics<1> [width=\textwidth,keepaspectratio] {sos-arch21}
	\includegraphics<2> [width=\textwidth,keepaspectratio] {sos-arch22}
	\includegraphics<3> [width=\textwidth,keepaspectratio] {sos-arch23}
	\includegraphics<4> [width=\textwidth,keepaspectratio] {sos-arch24}
	\includegraphics<5> [width=\textwidth,keepaspectratio] {sos-arch25}
	\includegraphics<6> [width=\textwidth,keepaspectratio] {sos-arch26}
	\includegraphics<7> [width=\textwidth,keepaspectratio] {sos-arch27}
    \end {center}
\end {frame}


%%%%%%%%%%%%%%%%%%%%%%%%%%%%%%%%%%%%%%%%%%%%%%%%%%%%%%%%%%%%%%%%%%%%%%%%%%%%%%
\section {Conclusion}

\begin {frame} {Outline}
    \tableofcontents [current]
\end {frame}


\begin {frame} {Summary}
    REST:
    \begin {itemize}
	\item is the current trend for sensor networks applications
	\item needs far too many protocols \\
	    \imply and complexity
    \end {itemize}

    Our idea:
    \begin {itemize}
	\item view a sensor network as a distributed operating system
	    offering a REST interface \\
	    \imply each sensor is REST-addressable \\
	    \imply each sensor is ``on the Internet'' through REST
	\item build upon CoRE experience
    \end {itemize}

\end {frame}

\begin {frame} {Challenges}
    \begin {itemize}
	\item design and implement the SOS protocol
	\item design and implement the Sensor Operating System
	\item prove that our architecture is viable \\
	    \imply experiment with some use-cases
	\item prove that SOS architecture is more efficient \\
	    \imply compare times on identical hardware \\
	    \imply compare code size
    \end {itemize}

\end {frame}

\begin {frame} {Questions?}
    \includegraphics [width=\textwidth,keepaspectratio] {tomatoes}
\end {frame}

\end {document}
